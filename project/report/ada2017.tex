%
% File acl2014.tex
%
% Contact: giovanni.colavizza@epfl.ch
%%
%% Based on the style files for ACL-2013, which were, in turn,
%% Based on the style files for ACL-2012, which were, in turn,
%% based on the style files for ACL-2011, which were, in turn, 
%% based on the style files for ACL-2010, which were, in turn, 
%% based on the style files for ACL-IJCNLP-2009, which were, in turn,
%% based on the style files for EACL-2009 and IJCNLP-2008...

%% Based on the style files for EACL 2006 by 
%%e.agirre@ehu.es or Sergi.Balari@uab.es
%% and that of ACL 08 by Joakim Nivre and Noah Smith

\documentclass[11pt]{article}
\usepackage{acl2014}
\usepackage{times}
\usepackage{url}
\usepackage{latexsym}

%\setlength\titlebox{5cm}

% You can expand the titlebox if you need extra space
% to show all the authors. Please do not make the titlebox
% smaller than 5cm (the original size); we will check this
% in the camera-ready version and ask you to change it back.


\title{A study on Amazon's distinctive features}

\author{Raphael Barman \\
  {\tt raphael.barman@epfl.ch} \\\And
  Hakim Invernizzi \\
  {\tt hakim.invernizzi@epfl.ch} \\\And
Albane Descombes \\
{\tt albane.descombes@epfl.ch} \\}

\date{}

\begin{document}
\maketitle
\begin{abstract}
  This document describes and discusses the results obtained in the context of the authors' project in ADA2017. The project aims at deepening the understanding 
  of Amazon's framework. The analysis is centred around three different aspects . \\ First, the review feature, with the goal to describe its usage from the userbase and answer the question of whether reviews are generally biased or not. \\ Second, the adverting feature which is composed of the 'also bought', 'also viewed' and 'bought together' features. The intra-correlation among the sub-features as well as their impact on the whole system are the subject of study. \\ Lastly, the categorisation feature and its impact on the user perception of a product. \\  The three features represent three different starting points for the analysis which should ideally converge in a single conclusion.
\end{abstract}


\section{Introduction}
  Explain the relevant information about metadata here.
  
\section{Review feature}
 - describing reviews:
   is there a relation between the length of the review and the product rating of the review?
   is there a relation between the length of the review and the helpfullness?
   is there a relationship between the helpfulness of a review and the product ranking of the review? 0.25! --> "good" reviews are more useful?
   
   -->bias from the reviewers
   -- look at the avg number of reviews per user!
   -- few reviewers have written many reviews!
   -- mean rating is high!
   -- variable standard deviation (kinda low in general)
   
   --> brand love/hate
   -- low standard deviation, high rating mean --> smells like bias
   -- do other reasearch connected to brand: which are the most popular brands? 
   - is there a relation between the number of reviews a product has and the product position in the ranking?
   - is there a relation between the average helpfulness of the reviews a product has and the product position in the ranking?
   - is there a relation between the average rating of a product has and the product position in the ranking?

\section{Advertising features}
  - spearman correlation of bought together and product ranks 0.40! extremely relevant
  - the other two with respect to product rank are weak
  - examine intra-correlation

\section{Categorisation feature}
  - wider categories have more reviews 
  - examine correlation category with overall rank/ helpfulness

\section{Conclusions}
 - try to sum up things here 
 - --> there's bias because people vote when they're happy and don't when they're unhappy!

\begin{thebibliography}{}

\bibitem[\protect\citename{Aho and Ullman}1972]{Aho:72}
Alfred~V. Aho and Jeffrey~D. Ullman.
\newblock 1972.
\newblock {\em The Theory of Parsing, Translation and Compiling}, volume~1.
\newblock Prentice-{Hall}, Englewood Cliffs, NJ.

\end{thebibliography}

\end{document}
